\section{Refresher}

This section is for
if your computer is already set up,
the measurement routine
is adapted to your needs,
and you simply want to refresh
your memory how to get (re)started.

\begin{enumerate}
  \item record a measurement with \code{routine\_measurement.py} \\
         i.e. adjust the variables \code{samplename}, 
         \code{path\_to\_meas\_folder}, \code{heatsink\_start},
         \code{pump\_end}, \code{use\_spectrometer},
         \code{shuffle\_pump} \\
         (as explained in section~\ref{sec:routine})
  \item calibrate the setup with \code{routine\_calibration.py} \\
         If you still have a valid calibration you can skip this.
         But if you didn't measure for a while, you probably want to recalibrate!
         (as explained in section~\ref{sec:calib})
  \item evaluate the measurement with \code{light\_light\_to\_file.py}
         specify the path to the measurement logfile \code{logfile}
         and the path to the calibration data \code{calib\_folder}.
         This script returns a file that you can open for example with Excel;
         the top row states all sorts of information,
         what were the measurement conditions,
         and what the columns mean at all.
\end{enumerate}

